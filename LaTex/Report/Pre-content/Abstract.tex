%!TEX root = ../Main.tex

\chapter*{Abstract}
This report documents the bachelor's project at the Aarhus University College of Engineering, entitled "' Safe Water in Buildings"'. The project was carried out by two electronics engineering (E) and an information and communication technology engineering (ICT) students, in collaboration with the company Grundfos, during the period February 1, 2019 - May 29, 2019 under the guidance of Carl Jakobsen.

The framework for the project is based on a problem, prepared by the project group in collaboration with Grundfos.

The project deals with the design and implementation of a prototype for a intelligent home water purification system, for the purification of water in smaller buildings (1-2 families). In addition, it must be possible for a user to have their own system built in exactly the way they want, so that their personal needs are met, therefore the system must be designed to allow this dynamic structure.

The realization of the project takes place through a "'Windows Presentation Forms"' (WPF) project, which functions as a graphical interface (GUI) for the system, as well as a \CS-project that acts as controller for the system. Through the system's GUI, the user can access all functionalities of the system and read data from it. In addition, a "'Programmable Logic Controller"' (PLC) has been used as an interface to sensors associated with the system.

The development and design process was based on a modified version of the project management tool SCRUM. The group works iteratively, in sprints of about a week's duration, thereby ensuring a manageable or efficient work process.

The project culminated in a system that met the group's expectations.